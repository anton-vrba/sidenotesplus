
\documentclass[twoside,10pt,ragged]{article}
\PassOptionsToPackage{%
     a4paper,% landscape,%
     bindingoffset=3mm,%
     left=20mm,%
     textwidth=110mm,%
     marginparsep=10mm,%
     marginparwidth=55mm,%
     top=20mm,%
     bottom=20mm,%
     headsep=1\baselineskip,%
     footskip = 2\baselineskip,%
     includeall}   {geometry}%
\usepackage  {geometry}
\RequirePackage[utf8]{inputenc}\usepackage[T1]{fontenc}
\usepackage[alerton]{sidenotesplus}

\RequirePackage[svgnames,dvipsnames]{xcolor}
\usepackage{lipsum}
\usepackage{xcolor}
%\setlipsum{%
%  par-before = \begingroup\color{gray},
%  par-after = \endgroup,
%  sentence-before = \begingroup\color{gray},
%  sentence-after = \endgroup
%}

%auto generate the bib file
\usepackage{filecontents}
%
\begin{filecontents}{\jobname.bib}
@book{Tufte1990,
	author = {Edward R. Tufte},
	title = {Envisioning Information},
	publisher = {Graphics Press},
	year = {1990},
	isbn = {0-9613921-1-8}
}

@book{Tufte2006,
	author = {Edward R. Tufte},
	title = {Beautiful Evidence},
	year = {2006},
	publisher = {Graphics Press, {LLC}},
	isbn = {0-9613921-7-7}
}

@BOOK{bringhurst:2002,
    title = {{T}he {E}lements of {T}ypographic {S}tyle},
    publisher = {Hartley \& Marks Publishers},
    year = {2013},
    author = {Robert Bringhurst},
    series = {Version 4.0: 20th Anniversary Edition},
    address = {Point Roberts, WA, USA}
    }

@Article{Einstein_1905e,
  author    = {A. Einstein},
  journal   = {Annalen der Physik},
  title     = {Ist die Trägheit eines Körpers von seinem Energieinhalt abhängig?},
  year      = {1905},
  number    = {13},
  pages     = {639--641},
  volume    = {323},
  doi       = {10.1002/andp.19053231314},
  file      = {:Articles/Einstein_1905e - Does the Inertia of a Body Depend upon its
  Energy-Content_.pdf:PDF;:Articles/Einstein_1905e - Ist Die Trägheit Eines Körpers Von Seinem
  Energieinhalt Abhängig_.pdf:PDF},
  groups    = {Relativity},
  publisher = {Wiley},
}

\end{filecontents}
\usepackage{mwe}
% -- language: English --
%
\usepackage[english]{babel}
% -- biblatex --
\usepackage[backend=biber,style=nature]{biblatex} % xxx
% the .bib file with the references
\addbibresource{\jobname.bib}


\usepackage{listings}
\lstset{
basicstyle=\sffamily,
  lineskip=0pt,
  aboveskip= 3pt,
  belowskip= 0pt,
}

\usepackage{tabularx}
\usepackage{amsmath}
\usepackage{mathabx}
\usepackage{tikz}
\usetikzlibrary{calc}
\usepackage{xspace}

\captionsetup{font=small}  % Requires Package{caption} loaded in sidenoteplus




% Author info
\title{\textbf{\textsf{sidenotesplus}} Example Pages}
\author{Anton Vrba}

\date{	\today}

         \PassOptionsToPackage{osf,sc}{mathpazo}%
         \RequirePackage{mathpazo}
         \linespread{1.05} % a bit more for Palatino

\let\OldTexttt\texttt
\renewcommand{\texttt}[1]{\OldTexttt{\color{MidnightBlue}{#1}}}


\newcommand{\someimage}[3]{% Width, height, label
\begin{tikzpicture}[x=1pt,y=1pt]% 4x3
    \path [fill=black!25] (0,0) rectangle (#1,#2);
    \draw [thick,black!40]
        (0,0) -- (#1,#2)
        (#1,0) -- (0,#2)
        (0.5*#1,0) -- (0.5*#1,#2)
        (0,0.5*#2) -- (#1,0.5*#2)
    ;
    \path [draw,very thick] (0,0) rectangle (#1,#2);
     \node at (0.5*#1,0.5*#2) {\sffamily\Huge #3}
    ;
\end{tikzpicture}%
}
\newcommand \describe \paragraph


\begin{document}


	\maketitle
	
	\begin{abstract}
		\noindent Here we demonstrate the features of \textsf{sidenotesplus},
        a \LaTeX\xspace package to manage the margin notes, figures, tables and captions.
        Also body text can be extended into the margin for wide figures, tables and equation.
        Twoside symmetry is preserved. For biblatex users, routines for side references are
        provided.
	\end{abstract}


Please first read \textsf{sidenotesplus.pdf} for the descriptions and usage of this package.
This document served as a test platform while developing the package, and uses the standard
\verb"article" \LaTeX\xspace class. The above right margin note list the first view
lines\sidenote|-200pt|{%
\ttfamily\upshape\textbackslash documentclass[twoside,10pt]\textbraceleft
article\textbraceright\\
\textbackslash PassOptionsToPackage\textbraceleft\\
\makebox[2ex]{} a4paper,\\
\makebox[2ex]{} bindingoffset=3mm,\\
\makebox[2ex]{} left=20mm,\\
\makebox[2ex]{} textwidth=110mm,\\
\makebox[2ex]{} marginparsep=10mm,\\
\makebox[2ex]{} marginparwidth=55mm,\\
\makebox[2ex]{} top=20mm, bottom=20mm,\\
\makebox[2ex]{} headsep=1\textbackslash baselineskip,\\
\makebox[2ex]{} footskip = 2\textbackslash baselineskip,\\
\makebox[2ex]{} includeall\textbraceright   \textbraceleft geometry\textbraceright\\
\textbackslash usepackage  \textbraceleft geometry\textbraceright\\
\textbackslash usepackage[alerton]\textbraceleft sidenotesplus\textbraceright\\
} of the document preamble.

Here we have three \sidenote<-15pt>{\textsf{\upshape\textbackslash sidenote<-15pt>} Test up}%
\sidenote!Blue!{\textsf{\upshape\textbackslash sidenote!Blue!} Test colour}%
\sidenote|-12mm|{\textsf{\upshape\textbackslash sidenote|-12mm|} but cannot float past \textsuperscript d above}
 and the commas are inserted automatically between the text markers. But, if a line break
 is between the two \verb"\sidenote " commands, then that requires a \% sign before the line
 break.

In many environments the floating option fails thus the fixed option is used. Example, a side
note used in an equation:

\begin{equation} \label{eq:123}
  a=b\quad \text{see\sidenote<0pt>{test} }
\end{equation}
an was coded \verb+a=b\quad\text{see\sidenote<0pt>{test}}+. in \eqref{eq:123} Important here is
the option \verb/<0pt>/ with any valid length.


Side notes can be placed without \sidenotetext*|-20.pt|{A sidenotetext without a mark. Also
testing if the command \emph{sidepar} works.
\sidepar And here we have a new paragraph. And here we have a new paragraph. And here we have a
new paragraph} references by using the \verb"\sidenote*" option



Similar to \verb"\footnotemark"  and \verb"\footnotetext" the package provides macros
\verb"\sidenotemark", \verb"\sidenotetext" and \verb"\sidenotetextbefore" with the same option
set. In usage \verb"\sidenotemark" is followed by \verb"\sidenotetext", whereas
\verb"\sidenotetextbefore" is followed by \verb"\sidenotemark".  The side not is placed relative
to the \verb"\sidenotetext" commands.

  \sidecitet*{bringhurst:2002} expertise is in typography, and the famous expression $E=mc^2$ was
  first presented in this paper\sidecite{Einstein_1905e}.

~

\noindent  The above paragraph was coded:
  \begin{verbatim}
   \sidecitet*{bringhurst:2002} expertise is in typography,
   and the famous expression $E=mc^2$ was first presented
   in this paper\sidecite{Einstein_1905e}.
  \end{verbatim}




The command \verb"\sidealert" provides a temporary margin notes rendered in red or
by the user’s defined \verb"!colour!". The alert mark\sidealert{This paragraph needs to be
expanded} has zero width so it does not alter the main text layout
and is also rendered in colour The package option \verb"alerton" needs to
be specified in the document preamble, if omitted the alerts are not printed.

\newpage

\subsection*{Figure demonstration page A}
\begin{figure}[h]
\centering
    \someimage{0.75*\textwidth}{70pt}{A}%
        \caption{A: Short caption}
    \label{imageA}
\end{figure}

\lipsum[3][4-8]

\begin{figure}[h]
\centering
    \someimage{0.75*\textwidth}{70pt}{B}%
    \caption[A: Long caption] {A: \emph{Long caption} \lipsum[3][1-3]}
    \label{imageB}
\end{figure}
%
%
\begin{marginfigure}|-500pt|%
    \someimage{\marginparwidth}{70pt}{C}%%
    \margincaption {A: small rectangle in the margin.\label{rectangle1}}%
\end{marginfigure}%
%
%
\begin{margintable}|-500pt|
\upshape
\begin{tabularx}{\marginparwidth}{c X}
 \hline
 \multicolumn{2}{c}{Long Table Heading}\\
 Item&Description\\
 \hline
 one& The width of this column depends on the
 width of the table.\\
 \hline
 \end{tabularx}
 \vskip-1.8ex
 \margincaption{A: Some description \label{mtable1}}
\end{margintable}
%
%
\begin{marginfigure}|-500pt|%
    \someimage{\marginparwidth}{70pt}{D}%%
    \margincaption[A: Second margin]{A: Second margin figure with a very long label to take many
    lines.\label{rectangle2}}%
\end{marginfigure}%
%
%
\sidenote*|-40pt|{The gaps between the above floats can be increase by setting the length of {\upshape \textbackslash
\textsf{marginparpush} }, e.g {\upshape 15pt}. The effect is visible on next page}
\setlength{\marginparpush}{15pt}
%
%
\begin{figure}[h]
\raggedinner
    \sidecaption[A: Side caption]{A:  \emph{Side caption} with raggedinner command in figure
    ennvironment with a lengthy text \lipsum[7][4-5] }
    \label{imageD}
    \someimage{0.75*\textwidth}{70pt}{E}%
\end{figure}
%
%
\begin{figure*}[h]
\centering
    \someimage{\linewidth}{100pt}{Full width figure}%
    \caption[A: Full width] {\emph{Full width} \lipsum[12][1-5]}
    \label{imagefw}
\end{figure*}
\lipsum[3][4-6]\sidenote<-10pt>{\lipsum[3][1-2]}\par\lipsum[7][4-9]
\newpage

\subsection*{Figure demonstration page B}
\begin{figure}[h]
\centering
    \someimage{0.75*\textwidth}{70pt}{A}%
        \caption{B: Short caption}
    \label{imageA}
\end{figure}

\lipsum[3][4-8]

\begin{figure}[h]
\centering
    \someimage{0.75*\textwidth}{70pt}{B}%
    \caption[B: Long caption] {B: \emph{Long caption} \lipsum[3][1-3]}
    \label{imageB}
\end{figure}
%
%
\begin{marginfigure}|-500pt|%
    \someimage{\marginparwidth}{70pt}{C}%%
    \margincaption {B: small rectangle in the margin.\label{rectangle1}}%
\end{marginfigure}%
%
%
\begin{margintable}|-500pt|
\upshape
\begin{tabularx}{\marginparwidth}{c X}
 \hline
 \multicolumn{2}{c}{Long Table Heading}\\
 Item&Description\\
 \hline
 one& The width of this column depends on the
 width of the table.\\
 \hline
 \end{tabularx}
 \vskip-1.8ex
 \margincaption{B: Some description \label{mtable1}}
\end{margintable}
%
%
\begin{marginfigure}|-500pt|%
    \someimage{\marginparwidth}{70pt}{D}%%
    \margincaption[B: Second margin]{B: Second margin figure with a very long label to take many
    lines.\label{rectangle2}}%
\end{marginfigure}%
%
%
\sidenote*|-40pt|{Note the larger gaps between the above floats as {\upshape \textbackslash
\textsf{marginparpush} } was set to {\upshape 15pt}.}
%
%
\begin{figure}[h]
\raggedinner
    \sidecaption[B: Side caption]{B:  \emph{Side caption} with raggedinner command in figure
    ennvironment \lipsum[7][4-5]}
    \label{imageD}
    \someimage{0.75*\textwidth}{70pt}{E}%
\end{figure}
%
%
\begin{figure*}[h]
\centering
    \someimage{\linewidth}{100pt}{Full width figure}%
    \caption[B: Full width] {\emph{Full width} \lipsum[12][1-5]}
    \label{imagefw}
\end{figure*}
\lipsum[3][4-6]\sidenote<-10pt>{\lipsum[3][1-2]}\par\lipsum[7][4-9]
\newpage


\newpage
And a final check
\listoffigures
\listoftables



\end{document}
